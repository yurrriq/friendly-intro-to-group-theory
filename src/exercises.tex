\documentclass[12pt]{amsart}

\usepackage[utf8]{inputenc}
\usepackage[english]{babel}

\usepackage{geometry}
\geometry{a4paper}

\usepackage[counterclockwise]{rotating}

\usepackage{amssymb,ebproof,enumerate}

\usepackage{minted}
\usemintedstyle{emacs}
\setminted[gap]{bgcolor=gray!10,mathescape}

\def\Circled#1{{\Large \textcircled{\small \bf #1}}}

\def\exercise#1#2{%
  \inputminted{gap}{gap/#1/#2.g}%
  \inputminted{gap}{answer/#1/#2.txt}%
}

\def\inputproof#1#2{%
  \begin{proof}[Proof of \Circled{#2}]
    \input{proof/#1/#2.tex}
  \end{proof}%
}

\title[A Friendly Introduction to Group Theory]{%
  Exercises from \\
  {\sl A Friendly Introduction to Group Theory} \\
  by David Nash%
}

\author{Eric Bailey}
\date{November 26, 2017}

\begin{document}

\maketitle
\tableofcontents
\newpage

\section{Preliminaries}

\subsection{Sets}

\begin{enumerate}[{\bf 1.}]
\item
  \begin{enumerate}[(a)]
  \item \exercise{1.1}{1a} \newpage
  \item \exercise{1.1}{1b}
  \item \exercise{1.1}{1c}
  \end{enumerate} \newpage

\item
  \begin{sideways}
    \begin{minipage}{\textheight}
      \inputproof{1.1}{2}
      \inputproof{1.1}{4}
    \end{minipage}
  \end{sideways}
\item
  \begin{enumerate}[(a)]
  \item
    \begin{proof}
      \begin{prooftree*}
        \hypo{x \in (A \cup B) \cup C}
        \infer1{x \in A \cup B \lor x \in C}
        \infer1{x \in A \lor x \in B \lor x \in C}
        \infer1{x \in A \lor x \in B \cup C}
        \infer1{x \in A \cup (B \cup C)}
        \infer1{(A \cup B) \cup C \subseteq A \cup (B \cup C)}

        \hypo{x \in A \cup (B \cup C)}
        \infer1{x \in A \lor x \in (B \cup C)}
        \infer1{x \in A \lor x \in B \lor x \in C}
        \infer1{x \in A \cup B \lor x \in C}
        \infer1{x \in (A \cup B) \cup C}
        \infer1{A \cup (B \cup C) \subseteq (A \cup B) \cup C}

        \infer2{(A \cup B) \cup C = A \cup (B \cup C)}
      \end{prooftree*}
    \end{proof}
  \item
    \begin{proof}
      \begin{prooftree*}
        \hypo{x \in (A \cap B) \cap C}
        \infer1{x \in A \cap B \land x \in C}
        \infer1{x \in A \land x \in B \land x \in C}
        \infer1{x \in A \land x \in B \cap C}
        \infer1{x \in A \cap (B \cap C)}
        \infer1{(A \cap B) \cap C \subseteq A \cap (B \cap C)}

        \hypo{x \in A \cap (B \cap C)}
        \infer1{x \in A \land x \in (B \cap C)}
        \infer1{x \in A \land x \in B \land x \in C}
        \infer1{x \in A \cap B \land x \in C}
        \infer1{x \in (A \cap B) \cap C}
        \infer1{A \cap (B \cap C) \subseteq (A \cap B) \cap C}

        \infer2{(A \cap B) \cap C = A \cap (B \cap C)}
      \end{prooftree*}
    \end{proof}
  \item
    \begin{proof}
      \begin{prooftree*}
        \hypo{x \in A \setminus (A \setminus B)}
        \infer1{x \in A \land x \not \in (A \setminus B)}
        % \infer1{x \in A \land \neg (x \in A \land x \not \in B)}
        \infer1{x \in A \land (x \not \in A \lor x \in B)}
        \infer1{x \in A \land x \in B}
        \infer1{x \in A \cap B}
        \infer1{A \setminus (A \setminus B) \subseteq A \cap B}

        \hypo{x \in A \cap B}
        \infer1{x \in A \land x \in B}
        \infer1{x \in A \land x \not \in A \setminus B}
        \infer1{x \in A \setminus (A \setminus B)}
        \infer1{A \cap B \subseteq A \setminus (A \setminus B)}

        \infer2{A \setminus (A \setminus B) = A \cap B}
      \end{prooftree*}
    \end{proof}
  \end{enumerate}
\end{enumerate}

\end{document}
