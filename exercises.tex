\documentclass[12pt]{amsart}

\usepackage[utf8]{inputenc}
\usepackage[english]{babel}

\usepackage{geometry}
\geometry{a4paper}

\usepackage{enumerate}

\usepackage{minted}
\usemintedstyle{emacs}
\setminted[gap]{bgcolor=gray!10,mathescape}

\def\Circled#1{{\Large \textcircled{\small \bf #1}}}

\def\exercise#1#2{%
  \inputminted{gap}{gap/#1/#2.g}%
  \inputminted{gap}{answers/#1/#2.txt}%
}

\title{%
  Exercises from \\
  {\sl A Friendly Introduction to Group Theory} \\
  by David Nash%
}

\author{Eric Bailey}
\date{November 26, 2017}

\begin{document}

\maketitle
\tableofcontents
\newpage

\section{Preliminaries}

\subsection{Sets Exercises}

\begin{enumerate}[{\bf 1.}]
\item
  \begin{enumerate}[(a)]
  \item \exercise{1.1}{1a} \newpage
  \item \exercise{1.1}{1b}
  \item \exercise{1.1}{1c}
  \end{enumerate} \newpage
\item {\sl Proof.} \Circled{2} Let $x \in R \cup (S \cap T)$.
  By definition $x \in R$ or $x \in S \cap T$. \vskip 1em

  \underline{Case 1: $x \in R$}
  Since $x \in R$, it follows that $x \in R \cup S$ and $x \in R \cup T$.
  Thus we have $x \in (R \cup S) \cap (R \cup T)$.
  \vskip 1em

  \underline{Case 2: $x \in S \cap T$}
  By definition $x \in S$ and $x \in T$. It follows that $x \in R \cup S$ and
  $x \in R \cup T$. Thus we have $x \in (R \cup S) \cap (R \cup T)$.
  \vskip 1em

  Together, Case 1 and Case 2 demonstrate that if $x \in R \cup (S \cap T)$,
  then $x \in (R \cup S) \cap (R \cup T)$ as well, which proves that
  $R \cup (S \cap T) \subseteq (R \cup S) \cap (R \cup T)$.
  \vskip 1em

  For the other containment, suppose that $x \in (R \cup S) \cap (R \cup T)$.
  Then specifically $x \in R \cup S$ and $x \in R \cup T$. These imply
  respectively that $x \in R$ or $x \in S$ and that $x \in R$ or $x \in T$.
  Hence, $x \in R \cup (S \cap T)$ and we have
  $(R \cup S) \cap (R \cup T) \subseteq R \cup (S \cap T)$.
  By double containment we have shown that
  $R \cup (S \cap T) = (R \cup S) \cap (R \cup T)$ as desired. \qed
\end{enumerate}

\end{document}
